\documentclass[letterpaper,12pt,titlepage]{article}

\usepackage{graphicx}
\usepackage{amssymb}
\usepackage{amsmath}
\usepackage{amsthm}

\usepackage{alltt}
\usepackage{float}
\usepackage{color}
\usepackage{url}

\usepackage[framemethod=TikZ]{mdframed}
\mdfdefinestyle{MyFrame}{%
    innertopmargin=\baselineskip,
    innerbottommargin=\baselineskip,
    innerrightmargin=20pt,
    innerleftmargin=20pt,
    backgroundcolor=gray!5!white,
    splitbottomskip = 5mm,
    splittopskip = 5mm,
    skipabove=5mm}

\usepackage{listings}
\usepackage{xcolor}
\usepackage{color}
\usepackage[font=small,format=plain,labelfont=bf,up,textfont=it,up]{caption}

\usepackage{balance}
\usepackage[TABBOTCAP, tight]{subfigure}
\usepackage{enumitem}
\usepackage{pstricks, pst-node}

%%keeps text of subsection on same line
\usepackage{titlesec}
\titleformat{\subsection}[runin]
  {\normalfont\large\bfseries}{\thesubsection}{1em}{}

\usepackage{geometry}
\geometry{textheight=9in, textwidth=6.5in}

%random comment

\newcommand{\cred}[1]{{\color{red}#1}}
\newcommand{\cblue}[1]{{\color{blue}#1}}

\usepackage{hyperref}
\usepackage{geometry}

%% Code Listing Configuration
\renewcommand{\lstlistingname}{Code}
\lstset{ %
  %Some lang opts: C++, C, Java, Python, Matlab, TeX, HTML, SQL, Verilog, VHDL, make, ...
  basicstyle=\footnotesize\ttfamily , % the size of the fonts that are used for the code
  numbers=left,                       % where to put the line-numbers
  numberstyle=\scriptsize\color{darkgray}, % the style that is used for the line-numbers
  stepnumber=2,                       % the step between two line-numbers.
  numbersep=5pt,                      % how far the line-numbers are from the code
  backgroundcolor=\color{white},      % choose the background color. You must add \usepackage{color}
  showspaces=false,                   % show spaces adding particular underscores
  showstringspaces=false,             % underline spaces within strings
  showtabs=false,                     % show tabs within strings adding particular underscores
  frame=tb,                           % adds a frame around the code
  rulesepcolor=\color{gray},          % if not set, the frame-color may be changed on line-breaks within not-black text (e.g. commens (green here))
  tabsize=2,                          % sets default tabsize to 2 spaces
  captionpos=t,                       % sets the caption-position
  breaklines=true,                    % sets automatic line breaking
  breakatwhitespace=false,            % sets if automatic breaks should only happen at whitespace
  title=\lstname,                     % show the filename of files included with \lstinputlisting;
  keywordstyle=\color{blue},          % keyword style
  commentstyle=\color{dkgreen},       % comment style
  stringstyle=\color{mauve},          % string literal style
  escapeinside={\%*}{*)},             % if you want to add a comment within your code
  morekeywords={*,...}                % if you want to add more keywords to the set
  framexbottommargin=5pt,
}

\def\name{Drake Bridgewater \& Ryan Phillips}

%% The following metadata will show up in the PDF properties
\hypersetup{
  colorlinks = true,
  urlcolor = black,
  pdfauthor = {\name},
  pdfkeywords = {cs472 ``computer architecture'' clements ``chapter 3''},
  pdftitle = {CS 472: Homework 5},
  pdfsubject = {CS 472: Homework 5},
  pdfpagemode = UseNone
}

\begin{document}
\hfill \name

\hfill \today

\hfill CS 472 HW 5

\section*{$(9.2)$} Why do computers use cache memory?

\begin{mdframed}[style=MyFrame]
Because it's much faster than main memory, and if it's able to be used frequently, instead of other, slower memory, it increases the overall performance of a system.
\end{mdframed}

\section*{$(9.3)$} What is the meaning of temporal locality and spatial locality?

\begin{mdframed}[style=MyFrame]
Addresses are said to exhibit Temporal locality if they are accessed over and over within a short time span.

Spatial locality occurs when the addresses are clustered within the same region of memory. (Clements, pg. 561)
\end{mdframed}

\section*{$(9.4)$} From first principles, derive an expression for the speedup fatio of memory system with cache (assume the hit ration is h and the ratio of the main storage access time to cache access time to cache access time is k, where k<1). Assume that the system is an ideal system and that you don't have to worry about the effect of clock cycle times.

\begin{mdframed}[style=MyFrame]
\end{mdframed}

\section*{$(9.5)$} For the following systems, calculate the speedup ratio S in the cas $t_c$ is the access time of the cache memory, $t_m$ is the access time of the main store, and h is the hit ratio.

\subsection*{a} $t_m=70ns,~t_c=7ns,~h=0.9$
\begin{mdframed}[style=MyFrame]
\end{mdframed}
\subsection*{b} $t_m=60ns,~t_c=3ns,~h=0.9$
\begin{mdframed}[style=MyFrame]
\end{mdframed}
\subsection*{c} $t_m=60ns,~t_c=3ns,~h=0.8$
\begin{mdframed}[style=MyFrame]
\end{mdframed}
\subsection*{d} $t_m=60ns,~t_c=3ns,~h=0.97$
\begin{mdframed}[style=MyFrame]
\end{mdframed}

\section*{$(9.6)$} For the following ideal systems, calculate the hit ratio h required to achieve the stated speedup ratio S.

\subsection*{a} $t_m=60ns,~t_c=3ns,~S=1.1$
\begin{mdframed}[style=MyFrame]
\end{mdframed}
\subsection*{b} $t_m=60ns,~t_c=3ns,~S=2.0$
\begin{mdframed}[style=MyFrame]
\end{mdframed}
\subsection*{c} $t_m=60ns,~t_c=3ns,~S=5.0$
\begin{mdframed}[style=MyFrame]
\end{mdframed}
\subsection*{d} $t_m=60ns,~t_c=3ns,~S=15.0$
\begin{mdframed}[style=MyFrame]
\end{mdframed}

\section*{$(9.8)$} For the following system that use a clocked microprocessor, caculate the maximum speedup ratio you could expect to see as h approaches 100\%.

\subsection*{a} $t_{cyc}=20ns,~t_m=75ns,~t_c=15ns$
\begin{mdframed}[style=MyFrame]
\end{mdframed}
\subsection*{b} $t_{cyc}=20ns,~t_m=75ns,~t_c=25ns$
\begin{mdframed}[style=MyFrame]
\end{mdframed}
\subsection*{c} $t_{cyc}=10ns,~t_m=75ns,~t_c=15ns$
\begin{mdframed}[style=MyFrame]
\end{mdframed}

\section*{$(9.11)$} In a direct-mapped cache memory system, what is the meaning of the following terms.

\subsection*{a} Word
\begin{mdframed}[style=MyFrame]
Basic unit of memory. Defined by the system (i.e. 64-bit registers).
\end{mdframed}
\subsection*{b} Line
\begin{mdframed}[style=MyFrame]
A unit of storage that contains several words. It's location directly corresponds to a line in main memory.
\end{mdframed}
\subsection*{c} Set
\begin{mdframed}[style=MyFrame]
A series of lines. The size of a set is the same as the cache.
\end{mdframed}

\section*{$(9.12)$} How is data in main store mapped on to each of the following?

\subsection*{a} A direct-mapped cache
\begin{mdframed}[style=MyFrame]
\end{mdframed}
\subsection*{b} A fully associative cache
\begin{mdframed}[style=MyFrame]
\end{mdframed}
\subsection*{c} A set associative cache
\begin{mdframed}[style=MyFrame]
\end{mdframed}

\section*{$(9.17)$} What is cache coherency?

\begin{mdframed}[style=MyFrame]
The term cache coherence implies that the data in the various cache and main memories are all in step. (Clements, pg. 582,583)
\end{mdframed}

\section*{$(9.22)$} Why is it harder to design a data cache than an instruction cache?

\begin{mdframed}[style=MyFrame]
Because an instruction cache requires fewer features. For example: entries are never modified, and you don't have to worry about swapping out instructions, because the program doesn't change during the course of its execution.
\end{mdframed}

\section*{$(9.23)$} When a CPU writes to the cache, both the item in the cache and the corresponding item in the memory must be updated. If data is not in the cache, it must be fetched from memory and loaded in the cache. If $t_1$ is the time taken to reload the cache on a miss, show that the effective average access time of the memory system is given by

$t_{avg} = ht_{c} + (1 - h)t_{m} + (1 - h)t_{l}$.

\begin{mdframed}[style=MyFrame]
\end{mdframed}

\section*{$(9.26)$} A system has a level 1 cache and a level 2 cache. The hit rate of the level 1 cache is 90\%, and the hit rate of the level 2 cache is 80\%. An access to level 1 cache requires one cycle, an access to level 2 cache requires four cycles, and an access to main memory requires 50 cycles. What is the average access time?

\begin{mdframed}[style=MyFrame]
\end{mdframed}

\section*{$(9.28)$} In the context of multilevel caches, what is the difference between a local miss rate and a global miss rate?

\begin{mdframed}[style=MyFrame]
local miss rate is the rate of misses that occur for a specific cache. The global miss rate takes all of the caches into account.
\end{mdframed}

\section*{$(9.35)$} A 64-bit processor has a 8-MB, four-way set-associative cache with 32-byte lines. How is the address arranged in terms of set, line, and offset bits?

\begin{mdframed}[style=MyFrame]
\end{mdframed}

\section*{$(9.41)$} What are the fundamental differences between cache memory (as found in a CPU) and cache memory found in a hard disk drive?

\begin{mdframed}[style=MyFrame]
\end{mdframed}

\section*{$(9.42)$} What are the differences between write-back and write-through caches, and what are the implications for system performance?

\begin{mdframed}[style=MyFrame]
\end{mdframed}

\section*{$(9.43)$} A computer with 32-bit address architecture has a memory management system with single-level 4 KB page tables. How much memory space must be devoted to the page tables?

\begin{mdframed}[style=MyFrame]
\end{mdframed}

\section*{$(9.45)$} A computer runs an instruction set with the characteristics in the following table.

Arithmatic operations, 70\%, 1 cycle

Conditional operations, 15\%, 2 cycles

Load, 10\%, 2 cycles

Store, 5\%, 2 cycles

Hit rate, 95\%

Cost of a cache miss (read), 10 cycles

Write-through time, 5 cycles (writes to memory are not buffered)

What is the average number of cycles per instruction?

\begin{mdframed}[style=MyFrame]
\end{mdframed}

\section*{$(9.46)$} Consider the following code that accesses three values in memory scalar integers x and s, and an integer vector y[i]. What is the memory latency in clock cycles for a trip round the loop (after the first iteration)? Assume that the array is not cached and each new access to the array results in a miss.

The system has both L1 and L2 caches. The access time of the L1 cache is two cycles, the acess time of the L2 cache is 6 cycles and main memory has an access time of 50 cycles. In this case all memory and cache memory accesses take place in parallel.

\lstset{language=C}
\begin{lstlisting}
for (i = 0; i < 100; i++)
{
x = y[i];
s = s + x;
}
\end{lstlisting}


\begin{mdframed}[style=MyFrame]
\end{mdframed}

\section*{$(9.57)$}

A computer with a 24-bit address bus has a main memory of size 16 MB and a cache size of 64 KB. The wordlength is two bytes.

\subsection*{a} What is the address format for a direct-mapped cache with a line size of 32 words?
\begin{mdframed}[style=MyFrame]
\end{mdframed}
\subsection*{b} What is the address format for a fully associative cache with a line size of 32 words?
\begin{mdframed}[style=MyFrame]
\end{mdframed}
\subsection*{c} What is the adress format for a four-way set-associative cache with a line size of 16 words?
\begin{mdframed}[style=MyFrame]
\end{mdframed}



\bibliographystyle{plain}
\bibliography{writeup}
\end{document}
