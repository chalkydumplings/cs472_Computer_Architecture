\documentclass[letterpaper,12pt,titlepage]{article}

\usepackage{graphicx}
\usepackage{amssymb}
\usepackage{amsmath}
\usepackage{amsthm}

\usepackage{alltt}
\usepackage{float}
\usepackage{color}
\usepackage{url}

\usepackage[framemethod=TikZ]{mdframed}
\mdfdefinestyle{MyFrame}{%
    innertopmargin=\baselineskip,
    innerbottommargin=\baselineskip,
    innerrightmargin=20pt,
    innerleftmargin=20pt,
    backgroundcolor=gray!5!white,
    splitbottomskip = 5mm,
    splittopskip = 5mm,
    skipabove=5mm}

\usepackage{listings}
\usepackage{xcolor}
\usepackage{color}
\usepackage[font=small,format=plain,labelfont=bf,up,textfont=it,up]{caption}

\usepackage{balance}
\usepackage[TABBOTCAP, tight]{subfigure}
\usepackage{enumitem}
\usepackage{pstricks, pst-node}

%%keeps text of subsection on same line
\usepackage{titlesec}
\titleformat{\subsection}[runin]
  {\normalfont\large\bfseries}{\thesubsection}{1em}{}

\usepackage{geometry}
\geometry{textheight=9in, textwidth=6.5in}

%random comment

\newcommand{\cred}[1]{{\color{red}#1}}
\newcommand{\cblue}[1]{{\color{blue}#1}}

\usepackage{hyperref}
\usepackage{geometry}

%% Code Listing Configuration
\renewcommand{\lstlistingname}{Code}
\lstset{ %
  %Some lang opts: C++, C, Java, Python, Matlab, TeX, HTML, SQL, Verilog, VHDL, make, ...
  basicstyle=\footnotesize\ttfamily , % the size of the fonts that are used for the code
  numbers=left,                       % where to put the line-numbers
  numberstyle=\scriptsize\color{darkgray}, % the style that is used for the line-numbers
  stepnumber=2,                       % the step between two line-numbers.
  numbersep=5pt,                      % how far the line-numbers are from the code
  backgroundcolor=\color{white},      % choose the background color. You must add \usepackage{color}
  showspaces=false,                   % show spaces adding particular underscores
  showstringspaces=false,             % underline spaces within strings
  showtabs=false,                     % show tabs within strings adding particular underscores
  frame=tb,                           % adds a frame around the code
  rulesepcolor=\color{gray},          % if not set, the frame-color may be changed on line-breaks within not-black text (e.g. commens (green here))
  tabsize=2,                          % sets default tabsize to 2 spaces
  captionpos=t,                       % sets the caption-position
  breaklines=true,                    % sets automatic line breaking
  breakatwhitespace=false,            % sets if automatic breaks should only happen at whitespace
  title=\lstname,                     % show the filename of files included with \lstinputlisting;
  keywordstyle=\color{blue},          % keyword style
  commentstyle=\color{dkgreen},       % comment style
  stringstyle=\color{mauve},          % string literal style
  escapeinside={\%*}{*)},             % if you want to add a comment within your code
  morekeywords={*,...}                % if you want to add more keywords to the set
  framexbottommargin=5pt,
}

\def\name{Drake Bridgewater \& Ryan Phillips}

%% The following metadata will show up in the PDF properties
\hypersetup{
  colorlinks = true,
  urlcolor = black,
  pdfauthor = {\name},
  pdfkeywords = {cs472 ``computer architecture'' clements ``chapter 3''},
  pdftitle = {CS 472: Homework 5},
  pdfsubject = {CS 472: Homework 5},
  pdfpagemode = UseNone
}

\begin{document}
\hfill \name

\hfill \today

\hfill CS 472 HW 5

\section*{$(9.2)$} Why do computer use cache memory?

\begin{mdframed}[style=MyFrame]
\end{mdframed}

\section*{$(9.3)$} What is the meaning of temporal locality and spatial locality?

\begin{mdframed}[style=MyFrame]
\end{mdframed}

\section*{$(9.4)$} From first principles, derive an expression for the speedup fatio of memory system with cache (assume the hit ration is h and the ratio of the main storage access time to cache access time to cache access time is k, where k<1). Assume that the system is an ideal system and that you don't have to worry about the effect of clock cycle times.

\begin{mdframed}[style=MyFrame]
\end{mdframed}

\section*{$(9.5)$} For the following systems, calculate the speedup ratio S in the cas $t_c$ is the access time of the cache memory, $t_m$ is the access time of the main store, and h is the hit ratio. 
\subsection*{a} $t_m=70ns,~t_c=7ns,~h=0.9$
\begin{mdframed}[style=MyFrame]
\end{mdframed}
\subsection*{b} $t_m=60ns,~t_c=3ns,~h=0.9$
\begin{mdframed}[style=MyFrame]
\end{mdframed}
\subsection*{c} $t_m=60ns,~t_c=3ns,~h=0.8$
\begin{mdframed}[style=MyFrame]
\end{mdframed}
\subsection*{d} $t_m=60ns,~t_c=3ns,~h=0.97$
\begin{mdframed}[style=MyFrame]
\end{mdframed}

\section*{$(9.6)$} For the following ideal systems, calculate the hit ratio h required to achieve the stated speedup ratio S.
\subsection*{a} $t_m=60ns,~t_c=3ns,~S=1.1$
\begin{mdframed}[style=MyFrame]
\end{mdframed}
\subsection*{b} $t_m=60ns,~t_c=3ns,~S=2.0$
\begin{mdframed}[style=MyFrame]
\end{mdframed}
\subsection*{c} $t_m=60ns,~t_c=3ns,~S=5.0$
\begin{mdframed}[style=MyFrame]
\end{mdframed}
\subsection*{d} $t_m=60ns,~t_c=3ns,~S=15.0$
\begin{mdframed}[style=MyFrame]
\end{mdframed}

\section*{$(9.8)$} For the following system that use 

\begin{mdframed}[style=MyFrame]
\end{mdframed}

\section*{$(9.11)$} 

\begin{mdframed}[style=MyFrame]
\end{mdframed}

\section*{$(9.12)$} 

\begin{mdframed}[style=MyFrame]
\end{mdframed}

\section*{$(9.17)$} 

\begin{mdframed}[style=MyFrame]
\end{mdframed}

\section*{$(9.22)$} 

\begin{mdframed}[style=MyFrame]
\end{mdframed}

\section*{$(9.23)$} 

\begin{mdframed}[style=MyFrame]
\end{mdframed}

\section*{$(9.26)$} 

\begin{mdframed}[style=MyFrame]
\end{mdframed}

\section*{$(9.28)$} 

\begin{mdframed}[style=MyFrame]
\end{mdframed}

\section*{$(9.35)$} 

\begin{mdframed}[style=MyFrame]
\end{mdframed}

\section*{$(9.41)$} 

\begin{mdframed}[style=MyFrame]
\end{mdframed}

\section*{$(9.42)$} 

\begin{mdframed}[style=MyFrame]
\end{mdframed}

\section*{$(9.43)$} 

\begin{mdframed}[style=MyFrame]
\end{mdframed}

\section*{$(9.45)$} 

\begin{mdframed}[style=MyFrame]
\end{mdframed}

\section*{$(9.46)$} 

\begin{mdframed}[style=MyFrame]
\end{mdframed}

\section*{$(9.57)$} 

\begin{mdframed}[style=MyFrame]
\end{mdframed}


\bibliographystyle{plain}
\bibliography{writeup}
\end{document}
