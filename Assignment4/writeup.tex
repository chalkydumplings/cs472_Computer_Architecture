\documentclass[letterpaper,12pt,titlepage]{article}

\usepackage{graphicx}
\usepackage{amssymb}
\usepackage{amsmath}
\usepackage{amsthm}

\usepackage{alltt}
\usepackage{float}
\usepackage{color}
\usepackage{url}

\usepackage[framemethod=TikZ]{mdframed}
\mdfdefinestyle{MyFrame}{%
    innertopmargin=\baselineskip,
    innerbottommargin=\baselineskip,
    innerrightmargin=20pt,
    innerleftmargin=20pt,
    backgroundcolor=gray!5!white,
    splitbottomskip = 5mm,
    splittopskip = 5mm,
    skipabove=5mm}

\usepackage{listings}
\usepackage{xcolor}
\usepackage{color}
\usepackage[font=small,format=plain,labelfont=bf,up,textfont=it,up]{caption}

\usepackage{balance}
\usepackage[TABBOTCAP, tight]{subfigure}
\usepackage{enumitem}
\usepackage{pstricks, pst-node}

%%keeps text of subsection on same line
\usepackage{titlesec}
\titleformat{\subsection}[runin]
  {\normalfont\large\bfseries}{\thesubsection}{1em}{}

\usepackage{geometry}
\geometry{textheight=9in, textwidth=6.5in}

%random comment

\newcommand{\cred}[1]{{\color{red}#1}}
\newcommand{\cblue}[1]{{\color{blue}#1}}

\usepackage{hyperref}
\usepackage{geometry}

%% Code Listing Configuration
\renewcommand{\lstlistingname}{Code}
\lstset{ %
  %Some lang opts: C++, C, Java, Python, Matlab, TeX, HTML, SQL, Verilog, VHDL, make, ...
  basicstyle=\footnotesize\ttfamily , % the size of the fonts that are used for the code
  numbers=left,                       % where to put the line-numbers
  numberstyle=\scriptsize\color{darkgray}, % the style that is used for the line-numbers
  stepnumber=2,                       % the step between two line-numbers.
  numbersep=5pt,                      % how far the line-numbers are from the code
  backgroundcolor=\color{white},      % choose the background color. You must add \usepackage{color}
  showspaces=false,                   % show spaces adding particular underscores
  showstringspaces=false,             % underline spaces within strings
  showtabs=false,                     % show tabs within strings adding particular underscores
  frame=tb,                           % adds a frame around the code
  rulesepcolor=\color{gray},          % if not set, the frame-color may be changed on line-breaks within not-black text (e.g. commens (green here))
  tabsize=2,                          % sets default tabsize to 2 spaces
  captionpos=t,                       % sets the caption-position
  breaklines=true,                    % sets automatic line breaking
  breakatwhitespace=false,            % sets if automatic breaks should only happen at whitespace
  title=\lstname,                     % show the filename of files included with \lstinputlisting;
  keywordstyle=\color{blue},          % keyword style
  commentstyle=\color{dkgreen},       % comment style
  stringstyle=\color{mauve},          % string literal style
  escapeinside={\%*}{*)},             % if you want to add a comment within your code
  morekeywords={*,...}                % if you want to add more keywords to the set
  framexbottommargin=5pt,
}

\def\name{Drake Bridgewater \& Ryan Phillips}

%% The following metadata will show up in the PDF properties
\hypersetup{
  colorlinks = true,
  urlcolor = black,
  pdfauthor = {\name},
  pdfkeywords = {cs472 ``computer architecture'' clements ``chapter 3''},
  pdftitle = {CS 472: Homework 4},
  pdfsubject = {CS 472: Homework 4},
  pdfpagemode = UseNone
}

\begin{document}
\hfill \name

\hfill \today

\hfill CS 472 HW 4

\section*{$(3.8)$} What are the relative advantages and disadvantages of general-purpose registers compared to separate address and data registers?

\begin{mdframed}[style=MyFrame]
\end{mdframed}

\section*{$(3.9)$} What is a misaligned operand? Why are misaligned operands such a problem in programming?

\begin{mdframed}[style=MyFrame]
A misaligned operand is just what it sounds like an operand that is not aligned causing the operand to span two cache lines either slowing the system down substantially or causing a bus error. From some articles online typically x86 and x64 will happily work with unaligned data but will tend to be slower where as on other processors the system will just error. From the ARM website "Non word-aligned load and store multiple, double, semaphore, synchronization, and coprocessor accesses always signal Data Abort with an Alignment fault status code when the U bit is set." \cite{ARM}
\end{mdframed}

\section*{$(3.24)$} What is the meaning of each of the P,U,B,W, and L bits in the encoding of an ARM memory reference instruction?
\begin{mdframed}[style=MyFrame]
\begin{tabular}{l l}
\textbf{P} & Do you want to adjust pointer before using?\\
\textbf{U} & Do you want to decrement the pointer instead of incrementing it?\\
\textbf{B} & Is this word access?\\
\textbf{W} & Do you want to update the pointer after use?\\
\textbf{L} & Do you want to store data in memory?
\end{tabular}
\end{mdframed}

\section*{$(3.26)$} What is the effect of LDR r0, [r5, r6, LSL r2] ?

\begin{mdframed}[style=MyFrame]
\colorbox{red}{
  
Whatever is at the address located specified by: address r5 + (address r6)*2^r2 is loaded into r0.


}
\end{mdframed}

\section*{$(3.30)$} What is the meaning of sign-extension in the context of copying data from one location to another?

\begin{mdframed}[style=MyFrame]
\end{mdframed}

\section*{$(3.33)$} Most RISC processors do not include a block move instruction. What are the advantages and disadvantages of the ARM's LDM and STM instructions?

\begin{mdframed}[style=MyFrame]
\end{mdframed}

\newpage
\section*{$(3.34)$} What is the effect of executing STMIB r13!,\{r0-r2,r4\}? Draw a picture of the state of the stack pointed at by r13 before and after this operation.
\begin{mdframed}[style=MyFrame]

The instruction STMIB is the Store Multiple Registers instruction and it increments the address before each transfer and with the addition of the exclimation mark on Rd the final address is written back to Rd \\

\begin{center}
\begin{tabular}{| c | c | c |}
\hline
& Before & After\\ \hline \hline
R0 & 0x00000000  &\\ \hline
R1 & 0x11111111  &\\ \hline
R2 & 0x22222222  &\\ \hline
R3 & 0x33333333  &\\ \hline
R4 & 0x44444444  &\\ \hline
R5 & 0x55555555  &\\ \hline
R6 & 0x66666666  &\\ \hline
R8 &  &\\ \hline
R9 &  &\\ \hline
R10 &  &\\ \hline
R11 &  &\\ \hline 
R12 &  &\\ \hline
R13 &  &\\ \hline
R14 &  &\\ \hline
R15 & PC&\\ \hline
\end{tabular}
\end{center}
\begin{verbatim}
			AREA test, CODE, READWRITE	
			ENTRY
			
			LDR 	r0, =0x00000000
			LDR		R1, =0x11111111
			LDR		R2, =0x22222222
			LDR		R3, =0x33333333
			LDR		R4, =0x44444444
			LDR		R5, =0x55555555
			LDR		R6, =0x66666666
			
			ADR 	R13, Stack
			STMIB r7!,{r0-r2,r4}
			SPACE 20
Stack		SPACE 20
			END
\end{verbatim}
\end{mdframed}

\newpage
\section*{$(3.36)$} Without using the ARM's multiplication instruction, write one or more instructions (using ADD, SUB, and shifting) to multiply by the following integers.

\begin{enumerate}[label=\Alph*]
\item 33
\item 1025
\item 4095
\end{enumerate}

\begin{mdframed}[style=MyFrame]
In binary: \\
\begin{tabular}{l l }
33 & 0010 0001 \\
1025 & 0100 0000 0001 \\
4096 & 1111 1111 1111 \\ \\
$33\times 1025 = 33825$ & 1000 0100 0010 0001
\end{tabular}

\begin{verbatim}
				LDR		R1, #0x21
				LDR		R2, #0x401
				LDR 	R3, #0x1000
				
				;if bit 
				 
                LSL     R1, 1           ;Multiply R1 by two
                MOV     R2, R1          ;Save 2*R1 for later
                LSL     R1, 1           ;Multiply R1 by four
                LSL     R1, 1           ;Multiply R1 by eight
                ADD     R1, R2          ;Add in 2*R1 to get 10*R1
\end{verbatim}
\end{mdframed}

\section*{$(3.44)$} What does the following code do?
\begin{verbatim}
TEQ     r0, #0
RSBMI   r0, r0, #0
\end{verbatim}

\begin{mdframed}[style=MyFrame]
\textbf{TEQ} is the instruction test equivalence it uses a bitwise exclusive OR operation when it is complete it sets the flags and the result is discarded unlike the EORS instruction\\
\textbf{RSBMI} is the instruction Reverse Subtract without carry but the addition of MI means that it will only execute when N status register is set, it is negative. 

Depending on if the value in R0 the N flag will get set accordingly and if that flag does get set then the reverse subtract operation will execute 
\end{mdframed}

\section*{$(3.48)$} What, in the context of assembly language, is a psuedo-operation?

\begin{mdframed}[style=MyFrame]
\end{mdframed}

\section*{$(3.54)$} Explain what this fragment of code does instruction by instruction and what purpose it achieves (assuming that register r0 is the register of interest). Note that the data in r0 must not be 0 on entry.
\begin{verbatim}
        MOV        r1,#0
loop    MOVS       r0,r0,LSL #1
        ADDCC      r1,r1,#1
        BCC        loop
\end{verbatim}

\begin{mdframed}[style=MyFrame]
\end{mdframed}

\newpage
\section*{$(3.60)$} A computer has three eight-element vectors in memory, Va, Vb, and Vc. Each element of a vector is a 32-bit word. Write the code to calculate all elements of Vc if the ith element is given by: $Vc_i = \frac{1}{2} ( Va_i + Vb_i)$
\begin{mdframed}[style=MyFrame]
\lstinputlisting[linerange=1-40]{../Lab3/quad_func/quad_func.asm}
\end{mdframed}
\newpage

\section*{$(3.61)$} Register r15 is the program counter. You can use it with certain instructions such as a MOV (e.g., MOV pc, r14). However, r15 cannot be used in conjunction with most data processing instructions. Why?

\begin{mdframed}[style=MyFrame]
\end{mdframed}

\bibliographystyle{plain}
\bibliography{writeup}
\end{document}
