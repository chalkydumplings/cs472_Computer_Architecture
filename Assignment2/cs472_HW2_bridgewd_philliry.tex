\documentclass[letterpaper,12pt,titlepage]{article}

\usepackage{graphicx}                                        
\usepackage{amssymb}                                         
\usepackage{amsmath}                                         
\usepackage{amsthm}                                          

\usepackage{alltt}                                           
\usepackage{float}
\usepackage{color}
\usepackage{url}

\usepackage{balance}
\usepackage[TABBOTCAP, tight]{subfigure}
\usepackage{enumitem}
\usepackage{pstricks, pst-node}

\usepackage{geometry}
\geometry{textheight=9in, textwidth=6.5in}

%random comment

\newcommand{\cred}[1]{{\color{red}#1}}
\newcommand{\cblue}[1]{{\color{blue}#1}}

\usepackage{hyperref}
\usepackage{geometry}

\def\name{Drake Bridgewater \& Ryan Phillips}

%% The following metadata will show up in the PDF properties
\hypersetup{
  colorlinks = true,
  urlcolor = black,
  pdfauthor = {\name},
  pdfkeywords = {cs472 ``computer architecture'' clements ``chapter 2''},
  pdftitle = {CS 472: Homework 2},
  pdfsubject = {CS 472: Homework 2},
  pdfpagemode = UseNone
}

\begin{document}
\hfill \name

\hfill \today

\hfill CS 472 HW 2

\begin{enumerate}

\item[$(2.5)$] Calculations can be performed to a precision of 0.001\%. How many bits does this require?
  
answer here

\item[$(2.13)$] Perform the following calculations in the stated bases.

\begin{verbatim}

a.
00110111_{2}
01011011_{2}

b.
00111111_{2}
01001001_{2}

c.
00120121_{16}
0A015031_{16}

d.
00ABCD1F_{16}
0F00800F_{16}

\end{verbatim}
  
answer here

\item[$(2.14)$] What is arithmatic overflow? When does it occur and how can it be detected?
  
answer here

\item[$(2.16)$] Convert 1234.125 into 32-bit IEEE floating-point format.
  
answer here

\item[$(2.17)$] What is the decimal equivalent of the 32-bit IEEE floating-point value CC4CC0000
  
answer here

\item[$(2.22)$] What is the difference between a \textit{truncation error} and a \textit{rounding} error? 
  
answer here

\item[$(2.40)$] Draw a truth table for the circuit in Figure P2.40 and explain what it does.
  
answer here

\item[$(2.45)$] It is possible to have n-input AND, OR, NAND, and NOR gates, where n\textgreater 2. Can you have an n-input XOR gate for n\textgreater 2? Explain your answer with a truth table.
  
answer here

\end{enumerate}

\bibliographystyle{plain}
\bibliography{cs472_HW2_bridgewd_philliry}
\end{document}

