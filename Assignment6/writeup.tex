\documentclass[letterpaper,12pt,titlepage]{article}

\usepackage{graphicx}
\usepackage{amssymb}
\usepackage{amsmath}
\usepackage{amsthm}

\usepackage{alltt}
\usepackage{float}
\usepackage{color}
\usepackage{url}

\usepackage[framemethod=TikZ]{mdframed}
\mdfdefinestyle{MyFrame}{%
    innertopmargin=\baselineskip,
    innerbottommargin=\baselineskip,
    innerrightmargin=20pt,
    innerleftmargin=20pt,
    backgroundcolor=gray!5!white,
    splitbottomskip = 5mm,
    splittopskip = 5mm,
    skipabove=5mm}

\usepackage{listings}
\usepackage{xcolor}
\usepackage{color}
\usepackage[font=small,format=plain,labelfont=bf,up,textfont=it,up]{caption}

\usepackage{balance}
\usepackage[TABBOTCAP, tight]{subfigure}
\usepackage{enumitem}
\usepackage{pstricks, pst-node}

%%keeps text of subsection on same line
\usepackage{titlesec}
\titleformat{\subsection}[runin]
  {\normalfont\large\bfseries}{\thesubsection}{1em}{}

\usepackage{geometry}
\geometry{textheight=9in, textwidth=6.5in}

%random comment

\newcommand{\cred}[1]{{\color{red}#1}}
\newcommand{\cblue}[1]{{\color{blue}#1}}

\usepackage{hyperref}
\usepackage{geometry}

%% Code Listing Configuration
\renewcommand{\lstlistingname}{Code}
\lstset{ %
  %Some lang opts: C++, C, Java, Python, Matlab, TeX, HTML, SQL, Verilog, VHDL, make, ...
  basicstyle=\footnotesize\ttfamily , % the size of the fonts that are used for the code
  numbers=left,                       % where to put the line-numbers
  numberstyle=\scriptsize\color{darkgray}, % the style that is used for the line-numbers
  stepnumber=2,                       % the step between two line-numbers.
  numbersep=5pt,                      % how far the line-numbers are from the code
  backgroundcolor=\color{white},      % choose the background color. You must add \usepackage{color}
  showspaces=false,                   % show spaces adding particular underscores
  showstringspaces=false,             % underline spaces within strings
  showtabs=false,                     % show tabs within strings adding particular underscores
  frame=tb,                           % adds a frame around the code
  rulesepcolor=\color{gray},          % if not set, the frame-color may be changed on line-breaks within not-black text (e.g. commens (green here))
  tabsize=2,                          % sets default tabsize to 2 spaces
  captionpos=t,                       % sets the caption-position
  breaklines=true,                    % sets automatic line breaking
  breakatwhitespace=false,            % sets if automatic breaks should only happen at whitespace
  title=\lstname,                     % show the filename of files included with \lstinputlisting;
  keywordstyle=\color{blue},          % keyword style
  commentstyle=\color{dkgreen},       % comment style
  stringstyle=\color{mauve},          % string literal style
  escapeinside={\%*}{*)},             % if you want to add a comment within your code
  morekeywords={*,...}                % if you want to add more keywords to the set
  framexbottommargin=5pt,
}

\def\name{Drake Bridgewater \& Ryan Phillips}

%% The following metadata will show up in the PDF properties
\hypersetup{
  colorlinks = true,
  urlcolor = black,
  pdfauthor = {\name},
  pdfkeywords = {cs472 ``computer architecture'' clements ``chapter 3''},
  pdftitle = {CS 472: Homework 6},
  pdfsubject = {CS 472: Homework 6},
  pdfpagemode = UseNone
}

\begin{document}
\hfill \name

\hfill \today

\hfill CS 472 HW 6

\section*{$(6.1)$} What is performance in the context of computer systems and why is it so difficult to define?

\begin{mdframed}[style=MyFrame]
Computer performance is difficult to define and is dependent on the slowest component. As our text book explains "computer's performance are used by the designer to improve the computers architecture and organization by locating bottlenecks and eliminating them or minimizing their effects." \cite{Clements}
\end{mdframed}

\section*{$(6.2)$} A system consists of a CPU, cache memory, main store, and hard disk drive. Where are time and effort best spent improving the system's performance? What factors affect your answer?

\begin{mdframed}[style=MyFrame]
The area engineers typically focus on for improving overall system performance is the slowest component and still today the slowest component is the HDD. The HDD has been replaced by SSD which are non-mechanical components allowing the system to be faster but does not allow for huge capacity storage. Therefore, computers are given this artificial appearance of speed by intelligent management of hardware that has unparalleled speed.
\end{mdframed}

\section*{$(6.4)$} A data transmission system transmits data in the the form of a master frame containing 16 sub-frames. Each sub-frame includes a 1024-bit data word and a 12-bit error-correcting code. The master frame itself contains a 32-bit error correcting code. What is the efficiency of this system?

\begin{mdframed}[style=MyFrame]
\end{mdframed}

\section*{$(6.6)$} Why is clock rate a poor metric of computer performance? What are the relative strengths and weaknesses of clock speed as a performance metric?

\begin{mdframed}[style=MyFrame]
\end{mdframed}

\section*{$(6.7)$} The timing diagram in Figure P6.7 illustrates a system in which operations occur as three consecutive clock cycles. Actions taking place in clock cycle 1 are scalalbe; that is, if the clock cycle time chances, the actions can be speeded up or slowed down correspondingly. In cycle 2, the action process 1 requires 25 ns and in clock cycle 3 the action process 2 requires 32 ns. If the clock cycle is less than the time required for process 1 or processes 2, then one ore more wait cycle have to be inserted for the process to complete.

What is the time to complete an operation if the clock cycle time is

a. 50 ns
b. 40 ns
c. 30 ns
d. 20 ns
e. 10 ns

\begin{mdframed}[style=MyFrame]
\end{mdframed}

%\section*{$(6.9)$} Can you think of a better metric than MIPS or clock speeds that gives a good impression of the power of a processor (without having to use benchmarks)?

\begin{mdframed}[style=MyFrame]
\end{mdframed}

\section*{$(6.11)$} Overlclocking a computer means operating at at a higher clock rate than that specified by its manufacturer; for example, a 2 GHz chip might be clocked at 2.1 GHz to squeeze more performance out of it.

Does overclocking disprove the famous aphorism ``There\'s no such thing as a free lunch," or is there a hidden cost? If so, what is the cost of overclocking?

\begin{mdframed}[style=MyFrame]
\end{mdframed}
\newpage
\section*{$(6.12)$} The following figures define the typical operating parameters of a processor.\\
\begin{center}
\begin{tabular}{l | c | c}
\textbf{Operation} & \textbf{Frequency} & \textbf{Cycles}\\ \hline
Arithmatic/logical instructions& 45\% & 1\\
Register load operations& 20\% & 3\\
Register store operations& 10\% & 2\\
All branch instructions& 25\% & 2 \\
\end{tabular}
\end{center}

If the clock rate could be reduced by 15\%, it would require only 2 cycles to perform a register load. Would that be a good idea?
\begin{mdframed}[style=MyFrame]

CPU~time(v1):\\
$=~(\# instructions)\times ((0.45*1)+(0.20*3)+(0.10*2)+(0.25*2))\times (clock)$\\
$=1\times(1.75)\times clock$ \\ \\
CPU~time(v2):\\
$=~(\# instructions)\times ((0.45*1)+(0.20*2)+(0.10*2)+(0.25*2))\times (clock \times 1.15)$\\
$=1\times(1.55)\times (1.15\times clock)$ \\
$=1\times(1.78)\times clock$ \\ \\
v2 is $\frac{1.78}{1.75} = 2\% $ slower
\end{mdframed}

\section*{$(6.13)$} A computer has the following parameters:\\
\begin{center}
\begin{tabular}{l | c | c}
\textbf{Operation} & \textbf{Frequency} & \textbf{Cycles}\\ \hline
Arithmatic/logical instructions& 45\% & 1\\
Register load operations& 20\% & 5\\
Register store operations& 10\% & 2\\
All branch instructions& 25\% & 8 \\
\end{tabular}
\end{center}

If the average performance of the computer (in terms of its CPI) is to be increased by 20\% while executing the same instruction mix, what target must be achieved for the cycles per conditional branch instruction?

\begin{mdframed}[style=MyFrame]
\end{mdframed}

\section*{$(6.14)$} A program is run on a computer with the following parameters.

Clock cycle time, 10 ns
Instructions with 1 cycle, 70\%
Instructions with 2 cycles, 20\%
Instructions with 3 cycles, 10\%

What is the MIPS rating of this computer?

\begin{mdframed}[style=MyFrame]
\end{mdframed}

\section*{$(6.16)$} In a particular system, a CPU is used for 78\% of the time and a disk drive for 22\% of the time. A designer has two options:

a. improve the disc performance by 40\% and the CPU performance by 20\%

b. improve the disc performance by 10\% and the CPU performance by 80\%

Which is the better option, and why?

\begin{mdframed}[style=MyFrame]
\end{mdframed}

\section*{$(6.17)$} For the following systems that have both serial and parallel activies, calculate the speedup ratio.

a. 10 processors, $f_s$ = 0.1
b. 100 processors, $f_s$ = 0.1
c. 5 processors, $f_s$ = 0.4
d. 100 processors, $f_s$ = 0.01

\begin{mdframed}[style=MyFrame]
\end{mdframed}

\section*{$(6.18)$} A system has a single core processor that costs \$150. Suppose that adding more cores to the chip costs \$10 per additional processor. (Note: For this system, the value of $f_s$ is 0.10).

If it is considered worthwhile adding cores until te incremental speedup ratio increases by less than 5\% over the orignal (unmodified) performance, what is the optimum number of processors? What percentag increase in cost is required to achieve this performance?

\begin{mdframed}[style=MyFrame]
\end{mdframed}

\section*{$(6.22)$} A coprocessor is added to a computer to speed up the execution time of string-processing instructions by a factor of 3.5. What fraction of the execution time must use these string-processing instructions in order to achieve an average speedup of 1.5?

\begin{mdframed}[style=MyFrame]
\end{mdframed}

\section*{$(6.25)$} A computer spends 25\% of its time accessing a hard disk. It spends 20\% of the time doing floating point. The hard disk is replaced by two disks operating in parallel and the floating point unit is replaced by one four times faster. The speed up is given by

equation here

So, the speedup for the disk is $S_{disk}$ - 2/2(2 * 0.75 + 1 - 0.75) = 1/1.75 = 1.429

The speedup for the floating-point unit is $S_{floating-point}$ = 4/(f * 0.80 + 1 - 0.80) = 4/1 = 4/3.4 = 1.176. The total speedup ratio is the product o the individual speedups which is 1.429 * 1.176 = 1.681. Is this answer correct?

\begin{mdframed}[style=MyFrame]
\end{mdframed}

\section*{$(6.26)$} Someone decided to use the following C code as part of a benchmark to determine the performance of a computer including its memory. It has two potential faults. What are they?

\begin{verbatim}
for (i = 0; i < 100; i++){
    p = q * s + 12345
    x = 0.0;
    for (j = 0; j < 60000; j++){
    x = x + A[j] * B[j];
    }
}
\end{verbatim}

\begin{mdframed}[style=MyFrame]
\end{mdframed}

\section*{$(6.31)$} For two benchmarks, x and y, show that their arithmatic mean is always higher than, or the sams as, the geometric mean.
>>>>>>> 6171dcda883b8a948b4ed3cdd705f134cc17d0d4

\begin{mdframed}[style=MyFrame]
\end{mdframed}

\bibliographystyle{plain}
\bibliography{writeup}
\end{document}
