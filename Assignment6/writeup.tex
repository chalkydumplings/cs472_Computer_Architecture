\documentclass[letterpaper,12pt,titlepage]{article}

\usepackage{graphicx}
\usepackage{amssymb}
\usepackage{amsmath}
\usepackage{amsthm}

\usepackage{alltt}
\usepackage{float}
\usepackage{color}
\usepackage{url}

\usepackage[framemethod=TikZ]{mdframed}
\mdfdefinestyle{MyFrame}{%
    innertopmargin=\baselineskip,
    innerbottommargin=\baselineskip,
    innerrightmargin=20pt,
    innerleftmargin=20pt,
    backgroundcolor=gray!5!white,
    splitbottomskip = 5mm,
    splittopskip = 5mm,
    skipabove=5mm}

\usepackage{listings}
\usepackage{xcolor}
\usepackage{color}
\usepackage[font=small,format=plain,labelfont=bf,up,textfont=it,up]{caption}

\usepackage{balance}
\usepackage[TABBOTCAP, tight]{subfigure}
\usepackage{enumitem}
\usepackage{pstricks, pst-node}

%%keeps text of subsection on same line
\usepackage{titlesec}
\titleformat{\subsection}[runin]
  {\normalfont\large\bfseries}{\thesubsection}{1em}{}

\usepackage{geometry}
\geometry{textheight=9in, textwidth=6.5in}

%random comment

\newcommand{\cred}[1]{{\color{red}#1}}
\newcommand{\cblue}[1]{{\color{blue}#1}}

\usepackage{hyperref}
\usepackage{geometry}

%% Code Listing Configuration
\renewcommand{\lstlistingname}{Code}
\lstset{ %
  %Some lang opts: C++, C, Java, Python, Matlab, TeX, HTML, SQL, Verilog, VHDL, make, ...
  basicstyle=\footnotesize\ttfamily , % the size of the fonts that are used for the code
  numbers=left,                       % where to put the line-numbers
  numberstyle=\scriptsize\color{darkgray}, % the style that is used for the line-numbers
  stepnumber=2,                       % the step between two line-numbers.
  numbersep=5pt,                      % how far the line-numbers are from the code
  backgroundcolor=\color{white},      % choose the background color. You must add \usepackage{color}
  showspaces=false,                   % show spaces adding particular underscores
  showstringspaces=false,             % underline spaces within strings
  showtabs=false,                     % show tabs within strings adding particular underscores
  frame=tb,                           % adds a frame around the code
  rulesepcolor=\color{gray},          % if not set, the frame-color may be changed on line-breaks within not-black text (e.g. commens (green here))
  tabsize=2,                          % sets default tabsize to 2 spaces
  captionpos=t,                       % sets the caption-position
  breaklines=true,                    % sets automatic line breaking
  breakatwhitespace=false,            % sets if automatic breaks should only happen at whitespace
  title=\lstname,                     % show the filename of files included with \lstinputlisting;
  keywordstyle=\color{blue},          % keyword style
  commentstyle=\color{dkgreen},       % comment style
  stringstyle=\color{mauve},          % string literal style
  escapeinside={\%*}{*)},             % if you want to add a comment within your code
  morekeywords={*,...}                % if you want to add more keywords to the set
  framexbottommargin=5pt,
}

\def\name{Drake Bridgewater \& Ryan Phillips}

%% The following metadata will show up in the PDF properties
\hypersetup{
  colorlinks = true,
  urlcolor = black,
  pdfauthor = {\name},
  pdfkeywords = {cs472 ``computer architecture'' clements ``chapter 3''},
  pdftitle = {CS 472: Homework 6},
  pdfsubject = {CS 472: Homework 6},
  pdfpagemode = UseNone
}

\begin{document}
\hfill \name

\hfill \today

\hfill CS 472 HW 6

\section*{$(6.1)$} What is computer performance and why is it so difficult to define?

\begin{mdframed}[style=MyFrame]
Computer performance is difficult to define and is dependent on the slowest component. As our text book explains "computer's performance are used by the designer to improve the computers architecture and organization by locating bottlenecks and eliminating them or minimizing their effects." \cite{Clements}
\end{mdframed}
\section*{$(6.2)$} 

\begin{mdframed}[style=MyFrame]
\end{mdframed}
\section*{$(6.4)$} 

\begin{mdframed}[style=MyFrame]
\end{mdframed}
\section*{$(6.6)$} 

\begin{mdframed}[style=MyFrame]
\end{mdframed}
\section*{$(6.7)$} 

\begin{mdframed}[style=MyFrame]
\end{mdframed}
\section*{$(6.9)$} 

\begin{mdframed}[style=MyFrame]
\end{mdframed}
\section*{$(6.11)$} 

\begin{mdframed}[style=MyFrame]
\end{mdframed}
\section*{$(6.12)$} 

\begin{mdframed}[style=MyFrame]
\end{mdframed}
\section*{$(6.13)$} 

\begin{mdframed}[style=MyFrame]
\end{mdframed}
\section*{$(6.14)$} 

\begin{mdframed}[style=MyFrame]
\end{mdframed}
\section*{$(6.16)$} 

\begin{mdframed}[style=MyFrame]
\end{mdframed}
\section*{$(6.17)$} 

\begin{mdframed}[style=MyFrame]
\end{mdframed}
\section*{$(6.18)$} 

\begin{mdframed}[style=MyFrame]
\end{mdframed}
\section*{$(6.22)$} 

\begin{mdframed}[style=MyFrame]
\end{mdframed}
\section*{$(6.25)$} 

\begin{mdframed}[style=MyFrame]
\end{mdframed}
\section*{$(6.26)$} 

\begin{mdframed}[style=MyFrame]
\end{mdframed}
\section*{$(6.31)$} 

\begin{mdframed}[style=MyFrame]
\end{mdframed}

\bibliographystyle{plain}
\bibliography{writeup}
\end{document}
