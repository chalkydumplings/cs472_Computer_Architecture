\documentclass[letterpaper,12pt,titlepage]{article}

\usepackage{graphicx}
\usepackage{amssymb}
\usepackage{amsmath}
\usepackage{amsthm}

\usepackage{alltt}
\usepackage{float}
\usepackage{color}
\usepackage{url}

\usepackage{listings}
\usepackage{xcolor}
\usepackage{color}
\usepackage[font=small,format=plain,labelfont=bf,up,textfont=it,up]{caption}

\usepackage{balance}
\usepackage[TABBOTCAP, tight]{subfigure}
\usepackage{enumitem}
\usepackage{pstricks, pst-node}

%%keeps text of subsection on same line
\usepackage{titlesec}
\titleformat{\subsection}[runin]
  {\normalfont\large\bfseries}{\thesubsection}{1em}{}

\usepackage{geometry}
\geometry{textheight=9in, textwidth=6.5in}

%random comment

\newcommand{\cred}[1]{{\color{red}#1}}
\newcommand{\cblue}[1]{{\color{blue}#1}}

\usepackage{hyperref}
\usepackage{geometry}

%% Code Listing Configuration
\renewcommand{\lstlistingname}{Code}
\lstset{ %
  %Some lang opts: C++, C, Java, Python, Matlab, TeX, HTML, SQL, Verilog, VHDL, make, ...
  basicstyle=\footnotesize\ttfamily , % the size of the fonts that are used for the code
  numbers=left,                       % where to put the line-numbers
  numberstyle=\scriptsize\color{darkgray}, % the style that is used for the line-numbers
  stepnumber=2,                       % the step between two line-numbers.
  numbersep=5pt,                      % how far the line-numbers are from the code
  backgroundcolor=\color{white},      % choose the background color. You must add \usepackage{color}
  showspaces=false,                   % show spaces adding particular underscores
  showstringspaces=false,             % underline spaces within strings
  showtabs=false,                     % show tabs within strings adding particular underscores
  frame=tb,                           % adds a frame around the code
  rulesepcolor=\color{gray},          % if not set, the frame-color may be changed on line-breaks within not-black text (e.g. commens (green here))
  tabsize=2,                          % sets default tabsize to 2 spaces
  captionpos=t,                       % sets the caption-position
  breaklines=true,                    % sets automatic line breaking
  breakatwhitespace=false,            % sets if automatic breaks should only happen at whitespace
  title=\lstname,                     % show the filename of files included with \lstinputlisting;
  keywordstyle=\color{blue},          % keyword style
  commentstyle=\color{dkgreen},       % comment style
  stringstyle=\color{mauve},          % string literal style
  escapeinside={\%*}{*)},             % if you want to add a comment within your code
  morekeywords={*,...}                % if you want to add more keywords to the set
  framexbottommargin=5pt,
}

\def\name{Drake Bridgewater \& Ryan Phillips}

%% The following metadata will show up in the PDF properties
\hypersetup{
  colorlinks = true,
  urlcolor = black,
  pdfauthor = {\name},
  pdfkeywords = {cs472 ``computer architecture'' clements ``chapter 3''},
  pdftitle = {CS 472: Homework 5},
  pdfsubject = {CS 472: Homework 5},
  pdfpagemode = UseNone
}
\begin{document}
\hfill \name

\hfill \today

\hfill CS 472 LAB 4

\section*{Summary of What Every Programmer Should Know About Memory article}
As the documented suggested I skipped to section 6 as other assignments need our effort too. The content is actually really interesting and I we will have to read it all at a later time, but it is intended to advise software developers "on how to write code which performs well in the various situations" \cite[p.~2]{Drepper} 

A CPU without a cache is rare as caches can cover up most of the cost of random access this is due to the advancements we have made in optimization of un-cached read and write. As a programmer performance is essential in this world of "I want now." Therefore as Drepper states "changes affected the level 1 cache... will likely yield the best results" \cite[p.~49]{Drepper} This can be achieved by aligning code and data and improving locality. This is easier said then done as optimization of L1i is done at the instruction level meaning that unless the programmer writes assembly then you are relaying on the compiler, but as a programmer you still can "indirectly determine the L1i use by guiding the compiler to create better code." \cite[p.~55]{Drepper} Some ways of creating better code as far as speed is to when a function is called a single time to ensure it is executed inline; when using gcc you can add the always\_inline function attribute at compilation time to instruct the compiler to move functions in with the code. This can become a problem the function is called multiple times as it will create many more instructions. For the next level of cache L2 and higher you want your code to dynamically adjust itself to the cache line size. 

Another way to hide latency, which is implemented on many of today's processors, is the use of pipelining, out-of-order execution, and prefetching. Prefetching can be triggered by certain
events (hardware prefetching) or explicitly requested by the program (software prefetching). \cite[p.~61]{Drepper} With software prefetching the programs don't have to change, but ensuring the access patterns don't happen across page boundaries is key. 

Multi-threading can also be optimized. There is concurrency optimization, atomicity optimization, and bandwidth considerations that 

\section*{Summary Memory Optimization article}

\bibliographystyle{plain}
\bibliography{writeup}
\end{document}
